\documentclass{CoursENSAM}

\begin{document}

%----------- Informations du rapport ---------

\ecole{École Nationale Supérieure d'Arts et Métiers Bordeaux} %Nom de l'ecole
\titre{Titre} %Titre du fichier
\UE{UE} %Nom de la UE
\sujet{Sujet} %Nom du sujet

\enseignant{\emph{\textbf{Enseignant :}}\\ Lagro $\sqrt{128}$ \textsc{Kromagn's} Bo220} %Nom de l'enseignant

%Commenter la ligne en-dessous si ce n'est pas un rapport de stage
\tuteur{\emph{\textbf{Encadrant entreprise :}}\\ Manip $\sqrt{40}$ \textsc{Cuirra's} Bo220} %Nom du tuteur de stage

\eleve{\emph{\textbf{Élèves :}}\\
L\&za  6-39 \textsc{Truffards} Bo220 \\
2'.rom's 88-173 \textsc{Rap's} Bo220} %Nom des élèves

%----------- Initialisation -------------------
        
\fairemargesintro %Afficher les marges
\fairepagedegarde %Créer la page de garde

%------------ Debut du rapport ----------------

%------------------- Table des matières --------------------------------
\renewcommand{\thesection}{\Roman{section}}
\renewcommand{\thesubsection}{\hspace*{0.1cm} \arabic{subsection}}
\renewcommand{\thesubsubsection}{\hspace*{0.2cm} \arabic{subsection}.\arabic{subsubsection}} %¨Permet d'obtenir la numérotation de section en lettres romaines
\DeclareTOCStyleEntries[
dynnumwidth]{tocline}{section,subsection,subsubsection} %Permet l'espacement entre le numéro et le nom de la (sous-)section


%------------------ Couleurs des sections & sous sections ----------------------------

%Couleur des titres de section (violetAM)
\sectionfont{\color{violetAM}{}}

%Couleur des titres de section (orangeAM)
\subsectionfont{\color{orangeAM}{}}

%%%%%%%%%%%%%%%%%%%%%%%%%%%%%%%%%%%%%%%%%%%%%%%%%%%%%%%%%%%%%%%%%%%%%%%%%%%%%%%%%%%%%%%%%%%%%%%%%%%%%%%%%%%%%%%%%%%%%%%%%
%\pagenumbering{roman} %Numerotation chiffre romains jusqu'a l'introduction

%\begin{abstract}
%    Si vous voulez mettre un abstract décommentez cette partie
%\end{abstract}
%\newpage


\renewcommand{\abstractname}{Remerciements}
\begin{abstract}
Merci Papa et Maman!
\end{abstract}

\newpage

%Commenter la ou les listes dont vous n'avez pas besoin
\tableofcontents
\listoffigures
\listoftables
\lstlistoflistings

%------------ Corps du rapport ----------------

\newpage
%\fairemarges
%\pagenumbering{arabic} %Début de numerotation avec nombres arabes

%penser a changer les input si vous changez les noms des dossiers
\input{Introduction/Intro}
\newpage
\input{Presentation de l'entreprise/Présentation}
\newpage
\input{Contenu du stage/Deroule du stage}
\newpage
\input{Conclusion/Conclusion}
\newpage
\input{Appendix/appendix}
\newpage

% Commenter les 3 lignes suivantes si pas de bibliographie
\bibliographystyle{unsrt}
\bibliography{bibs/export}
\newpage

\end{document}